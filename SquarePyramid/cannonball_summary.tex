\documentclass{article}
\usepackage{amsmath,amsfonts,amsthm}
\usepackage{xcolor,ulem,cancel}

\newtheorem{lemma}{Lemma}
\numberwithin{lemma}{section}
\newtheorem{corollary}[lemma]{Corollary}
\newtheorem{theorem}{Theorem}

\usepackage[letterpaper,left=2.5cm,right=2.5cm,
top=2.5cm,bottom=2.5cm]{geometry}

\renewcommand{\thesection}{\arabic{section}}
\renewcommand{\thesubsection}{\arabic{section} (\alph{subsection})}
\renewcommand{\thesubsubsection}{\thesection (\alph{subsection})(\roman{subsubsection})}

\newcommand{\changed}[1]{{\color{teal} #1}}

\title{Formal Proof of the Cannonball Problem}
\author{int-y1}

\begin{document}
\maketitle
%\pagenumbering{arabic}

The cannonball problem asks for all positive integers that are both square and square pyramidal.
Square numbers are of the form $x^2$, and square pyramidal numbers are of the form
$1^2 + 2^2 + \dots + x^2 = x(x+1)(2x+1)/6$.

This Lean 4 proof shows that the only positive integer solutions to $x(x+1)(2x+1) = 6y^2$ are
$(x, y) = (1, 1), (24, 70)$. As a corollary, $1$ and $4900$ are the only positive integers that are both
square and square pyramidal.

The Lean 4 proof follows W. S. Anglin's proof, which I have copy-pasted below. However, I changed parts of the proof to make it easier to translate to Lean 4. The additions are highlighted in \changed{teal} and the deletions contain a strikethrough. There are 2 notable changes that I made to Anglin's proof:
\begin{itemize}
\item In Lemma \ref{lem1}, I added the fact that $w/\gcd(x,y)$ needs to be an integer.
\item In Lemma \ref{lem8}, I rewrote the statement of this lemma.
\end{itemize}

\section{Solutions where $x$ is even}

Suppose, then, that $x$ is even. Under this assumption we can solve the equation with
the help of the following three lemmas which were probably all known to Fermat.

\begin{lemma}\label{lem1}
The area of a Pythagorean triangle is never a square.
\end{lemma}
\begin{proof}
Suppose, on the contrary, there are Pythagorean triangles with square
areas. Let $w^2$ be the smallest area for which such a triangle exists. Let $x$ and $y$ be
the legs of a Pythagorean triangle with area $w^2$. Then $x^2 + y^2 = z^2$ for some integer
$z$, and $xy/2 = w^2$.
\changed{Note that $x/\gcd(x,y)$ and $y/\gcd(x,y)$ are the legs of a Pythagorean triangle with area $(w/\gcd(x,y))^2$. Also note that $w/\gcd(x,y)$ is an integer because $\gcd(x,y)^2 \mid xy = 2w^2$, and $a^2 \mid 2b^2$ implies $a \mid b$.}
Since $w$ is minimal, $x$ and $y$ are relatively prime \changed{(i.e. $\gcd(x,y) = 1$).} \sout{and, without
loss of generality, we may take $x$ odd and $y$ even. (It follows by a congruence
consideration modulo $4$ that $x$ and $y$ are not both odd.)} By the well-known theorem
for Pythagorean triangles, there are relatively prime positive integers $r$ and $s$ of
different parity such that\changed{, without loss of generality,} $x = r^2 - s^2$ and $y = 2rs$. Hence $(r^2-s^2)rs = w^2$ and $s/4 < s \le w^2$. \changed{From $r > 0$, $s > 0$, and $x = r^2-s^2 > 0$, it follows that $r-s > 0$.} Since $r$, $s$, $r-s$ and $r+s$ are \changed{positive and} pairwise relatively prime, and since
$(r-s)(r+s)rs = w^2$, it follows that there are positive integers $a$, $b$, $c$ and $d$ such
that $r = a^2$, $s = b^2$, $a^2-b^2 = r-s = c^2$ and $a^2+b^2 = r+s = d^2$. \sout{Note that $c$
and $d$ are relatively prime since $r-s$ and $r+s$ are relatively prime.} Noting also
that $c$ and $d$ are odd (because $r$ and $s$ have different parity), let $X = (c+d)/2$
and $Y = (d-c)/2$. \changed{Then $X^2+Y^2 = a^2$. From $2XY = b^2$, it follows that $2 \mid b$, so $XY/2 = (b/2)^2 = s/4$ is a square.} \sout{Then $X$ and $Y$ are relatively prime and $X^2+Y^2 = a^2$. Hence
one of $X$ and $Y$ is even, and $XY/2 = (d^2-c^2)/8 = b^2/4 = s/4$ is a square
integer.} Since the triangle with sides $X$, $Y$ and $a$ is a Pythagorean triangle with
square area $s/4$, it follows from the minimality of $w^2$ that $w^2 \le s/4$. Contradiction.
\end{proof}

\changed{
\begin{corollary}\label{cor1}
Let $(x, y, z, w)$ be an integer solution to $x^2+y^2 = z^2$ and $xy/2 = w^2$. Then either $x = 0$ or $y = 0$.
\end{corollary}
\begin{proof}
For the sake of contradiction, suppose $x \ne 0$ and $y \ne 0$. If $x$ and $y$ have the same sign, then $(|x|, |y|)$ violates Lemma \ref{lem1}. If $x$ and $y$ have different signs, then $w^2 = xy/2 < 0$ has no solutions in $w$.
\end{proof}
}

\begin{lemma}\label{lem2}
There are no positive integers $x$ such that $2x^4+1$ is a square.
\end{lemma}
\begin{proof}
To obtain a contradiction, suppose that $(x, y)$ is the least positive integer
solution of $2x^4+1 = y^2$. Then for some positive integer $s$, $y = 2s+1$ and $x^4 = 2s(s+1)$. If $s$ is odd then $s$ and $2(s+1)$ are relatively prime and, for some
integers $u$ and $v$, $s = u^4$ and $2(s+1) = v^4$. This gives $2(u^4+1) = v^4$ \sout{with $u$ odd
and $v$ even. Hence we have $2(1+1) \equiv 0 \pmod 8$}. \changed{However, $2(u^4+1) \bmod 8 \in \{2, 4\}$ and $v^4 \bmod 8 \in \{0, 1\}$.} Since this is impossible, $s$ cannot
be odd. Since $s$ is even, $2s$ and $s+1$ are relatively prime, and there are integers $u$
and $v$, both greater than $1$, such that $2s = u^4$ and $s+1 = v^4$. Let $w$ be the positive
integer such that $u = 2w$. Let $a$ be the positive integer such that $v^2 = 2a+1$. Then
$u^4/2+1 = s+1 = v^4$ so that $2w^4 = (v^4-1)/4 = a(a+1)$. Since $v^2 = 2a+1$,
it follows from congruence considerations modulo $4$ that $a$ is even. Since $2w^4 = a(a+1)$,
it follows that there are positive integers $b$ and $c$ such that $a = 2b^4$ and
$a+1 = c^4$. However, this implies that $2b^4+1 = (c^2)^2$ and hence $y \le c^2$ (by the
minimality of $(x, y)$). On the other hand, $c^2 \le a+1 < v^2 \le s+1 < y$. Contradiction.
\end{proof}

\changed{
\begin{corollary}\label{cor2}
Let $(x, y)$ be an integer solution to $2x^4+1 = y^2$. Then $x = 0$.
\end{corollary}
}

\begin{lemma}\label{lem3}
There is exactly one positive integer $x$, namely, $1$, such that $8x^4+1$ is a square.
\end{lemma}
\begin{proof}
\changed{Firstly, $8x^4+1 \ne (2s)^2$ because $8x^4+1$ is odd and $(2s)^2$ is even.}
Suppose $8x^4+1 = (2s+1)^2$. Then $2x^4 = s(s+1)$. If $s$ is even then
there are integers $u$ and $v$ such that $s = 2u^4$ and $s+1 = v^4$. In that case,
$2u^4+1 = s+1 = v^4$ and, by \changed{Corollary \ref{cor2}} \sout{Lemma \ref{lem2}}, $u = 0$ and, hence, $x = 0$. If $s$ is odd
then there are integers $u$ and $v$ such that $s = u^4$ and $s+1 = 2v^4$. In
that case, $u^4+1 = 2v^4$\changed{, and so, $u$ is odd. Using $v^4 = (u^4+1)/2$ and $u = 2u'+1$, we have $(v^4-u^2)/2 = (2u'^2+2u')^2$ and $(v^4+u^2)/2 = (2u'^2+2u'+1)^2$.} \sout{Since $u$ is odd, a congruence consideration modulo $4$
shows that $v$ is odd. Squaring both sides of $u^4+1 = 2v^4$, we obtain $4v^8-4u^4 = u^8-2u^4+1$
and hence $(v^4-u^2)(v^4+u^2) = ((u^4-1)/2)^2$, an integer square.
Since $v^4$ and $u^2$ are relatively prime,} it follows that both $(v^4-u^2)/2$ and
$(v^4+u^2)/2$ are integer squares. Now $(v^2-u)^2 + (v^2+u)^2 = 4(v^4+u^2)/2 = A^2$ and $(v^2-u)(v^2+u)/2 = (v^4-u^2)/2 = B^2$. By \changed{Corollary \ref{cor1},} \sout{Lemma \ref{lem1}, this is impossible unless}
$v^2 = \pm u$. Since $u^4+1 = 2v^4$, we obtain $u^4-2u^2+1 = 0$ and $u^2 = 1$. From this
it follows that $s = 1$ and $x = \pm 1$.
\end{proof}

\changed{
\begin{lemma}\label{lem3a}
Suppose $x$ and $y$ are nonnegative and coprime. Suppose $xy/3$ is a square. Then $x \not\equiv 2 \pmod 3$.
\end{lemma}
\begin{proof}
Firstly, $3 \mid xy$. If $3 \mid x$ then $x \equiv 0 \pmod 3$. Otherwise, suppose $3 \mid y$ and $y = 3k$. Then $x$ and $k$ are coprime and $xk$ is a square. It follows that $x$ is a square, and therefore, $x \bmod 3 \in \{0, 1\}$.
\end{proof}
}

With the above three lemmas, we are now in a position to solve $x(x+1)(2x+1) = 6y^2$
under the assumption that $x$ is even. Suppose, then, that $x$ is even.
Then $x+1$ is odd. Since \changed{$x/2$} \sout{$x$}, $x+1$ and $2x+1$ are relatively prime in pairs, it
follows \changed{from Lemma \ref{lem3a} that} \sout{that $x+1$ and $2x+1$ (both being odd) are either squares or triples of
squares. Thus} $x+1 \not\equiv 2 \pmod 3$ and $2x+1 \not\equiv 2 \pmod 3$. Hence $x \equiv 0 \pmod 3$,
and for some nonnegative integers $p$, $q$ and $r$, we have $x = 6q^2$, $x+1 = p^2$ and
$2x+1 = r^2$. Thus $6q^2 = (r-p)(r+p)$. Since $p$ and $r$ are both odd, $4$ is a factor
of $(r-p)(r+p) = 6q^2$ and thus $q$ is even. Let $q'$ be the integer such that $q = 2q'$.
We now have $6q'^2 = ((r-p)/2)((r+p)/2)$ and, since $(r-p)/2$ and $(r+p)/2$
are relatively prime (because $r^2$ and $p^2$ are relatively prime), we obtain one of the
following two cases.

Case (i). One of $(r-p)/2$ and $(r+p)/2$ has the form $6A^2$ and the other has
the form $B^2$ (where \changed{$B$ is positive} \sout{$A$ and $B$ are nonnegative integers}). Then $p = \pm(6A^2-B^2)$
and \changed{$6q^2 = 4(6q'^2) = 4(6A^2B^2)$} \sout{$q = 2AB$}. Since $6q^2+1 = x+1 = p^2$, we have $24A^2B^2 + 1 = (6A^2-B^2)^2$ or
$(6A^2-3B^2)^2-8B^4 = 1$. By Lemma \ref{lem3}, \changed{$B = 1$} \sout{$B = 0$ or $1$} and, hence, \changed{$A^2 \in \{0, 1\}$ and} $x = 6q^2 = 0$ or $24$.
The only nontrivial solution is thus with $x = 24$.

Case (ii). One of $(r-p)/2$ and $(r+p)/2$ has the form $3A^2$ and the other has
the form $2B^2$ (where \changed{$B$ is positive} \sout{$A$ and $B$ are nonnegative integers}). Then $p = (3A^2-2B^2)$
and \changed{$6q^2 = 4(6q'^2) = 4(6A^2B^2)$} \sout{$q = 2AB$}. This gives $24A^2B^2+1 = (3A^2-2B^2)^2$ and hence
$(3A^2-6B^2)^2-2(2B)^4 = 1$. \changed{This contradicts Lemma \ref{lem2}.} \sout{By Lemma \ref{lem2}, $B = 0$ and hence $x = 6q^2 = 0$.}

Thus when $x$ is even, the only solution to Lucas's puzzle is $x = 24$ cannonballs
along the base of the square pyramid.

\section{Solutions where $x$ is odd}

We have solved Lucas's problem under the assumption that $x$ is even. In this
section we solve it under the assumption that $x$ is odd. To do this, we first
investigate the solutions of the Diophantine equation $X^2-3Y^2 = 1$.

Let $a = 2+\sqrt 3$ and $b = 2-\sqrt 3$. Note that $ab = 1$. Where $n$ is any nonnegative
integer, let $u_n = (a^n+b^n)/2$ and $v_n = (a^n-b^n)/(2\sqrt 3)$. Then $u_n$ and $v_n$ are
integers, and it is a well-known result (from the theory of the Pell equation) that
when $n \ge 1$, $(u_n, v_n)$ is the $n$th positive integer solution of $X^2-3Y^2 = 1$.
Of course, when $n = 0$, we have the solution $X = 1$ and $Y = 0$.

In order to show that $x = 1$ is the only odd positive integer such that
$x(x+1)(2x+1)$ has the form $6y^2$, we use the following lemmas.

\begin{lemma}\label{lem4}
Where $m$ and $n$ are nonnegative integers, $u_{m+n} = u_m u_n + 3v_m v_n$ and
$v_{m+n} = u_m v_n + u_n v_m$. Also if $m-n \ge 0$, then $u_{m-n} = u_m u_n - 3v_m v_n$ and $v_{m-n} = -u_m v_n + u_n v_m$.
\end{lemma}
\begin{proof}
This follows \changed{from results already in mathlib4 $\ddot\smile$} \sout{by straight calculation from the definitions of $u_n$ and $v_n$}.
\end{proof}

Using Lemma \ref{lem4}, it is not hard to obtain the following result.

\begin{lemma}\label{lem5}
Where $m$ is a nonnegative integer, $u_{m+2} = \changed{4}\cancel{2u_1} u_{m+1} - u_m$ and $v_{m+2} = \changed{4}\cancel{2u_1} v_{m+1} - v_m$.
\end{lemma}

Using the fact that $(u_m, v_m)$ is a solution of $X^2-3Y^2 = 1$, we also have the
following.

\begin{lemma}\label{lem6}
Where $m$ is a nonnegative integer, $u_{2m} = 2u_m^2-1 = 6v_m^2+1$ and
$v_{2m} = 2u_m v_m$.
\end{lemma}

\begin{lemma}\label{lem7}
Let $m$, $n$ and $r$ be nonnegative integers such that $2rm-n$ is nonnegative.
Then $u_{2rm \pm n} \equiv (-1)^r u_n \pmod{u_m}$.
\end{lemma}
\begin{proof}
Using mathematical induction on $r$ together with Lemma \ref{lem4}, we can show
that $u_{(2r+1)m} \equiv 0 \pmod{u_m}$ and $v_{2rm} \equiv 0 \pmod{u_m}$. Since, by Lemma \ref{lem6},
$u_{2rm} = 2u_{rm}^2-1 = 6v_{rm}^2+1$, it follows that $u_{2rm} \equiv (-1)^r \pmod{u_m}$. Thus
$u_{2rm \pm n} = u_{2rm} u_n \pm 3v_{2rm} v_n \equiv (-1)^r u_n \pmod{u_m}$ (using Lemma \ref{lem4}).
\end{proof}

Let us consider the first few values of $u_n$. Starting with $n = 0$, we have $1, 2, 7, 26, 97, 362$,
and so on. If we consider these values modulo $5$, we have $1, 2, 2, 1, 2, 2, \dots$.
By Lemma \ref{lem5} it follows that this sequence is periodic. If we consider the values of $u_n$
modulo $8$, we obtain $1, 2, 7, 2, 1, 2, \dots$. By Lemma \ref{lem5}, this is a purely periodic
sequence with period length $4$. Note that when $n$ is even, $u_n$ is odd. Using the laws
of quadratic reciprocity, the above comments lead us to the following two lemmas.

\begin{lemma}\label{lem8}
\sout{If $n$ is even then $u_n$ is an odd nonmultiple of $5$ and $\left(\dfrac{5}{u_n}\right) = 1$ iff $n$ is a multiple of $3$.} \changed{Suppose $n$ is even. Firstly, $u_n$ is odd. Secondly, $u_n$ is not a multiple of $5$. Thirdly, $\left(\dfrac{5}{u_n}\right) = -1$ iff $n$ is not a multiple of $3$. (This lemma was rewritten so that it can be used in Lemma \ref{lem10})}
\end{lemma}

\begin{lemma}\label{lem9}
If $n$ is even then $u_n$ is odd and $\left(\dfrac{-2}{u_n}\right) = 1$ iff $n$ is a multiple of $4$.
\end{lemma}

The following and final lemma was first proved by Ma.

\begin{lemma}\label{lem10}
Where $n$ is a nonnegative integer, $u_n$ has the form $4M^2+3$ only when
$u_n = 7$.
\end{lemma}
\begin{proof}
Suppose $u_n = 4M^2+3$. Then $u_n \equiv 3$ or $7 \pmod 8$ and, from the sequence
of values of $u_n$ modulo $8$, it follows that $n$ has the form $8k \pm 2$. \changed{If $k = 0$, then $n = 2$ and $u_n = 7$ and we're done. Otherwise, suppose $k > 0$ so that} \sout{Suppose}
$n \ne 2$ (and hence $u_n \ne 7$). Then we can write $n$ in the form $2k' 2^s \pm 2$ where $k'$ is
odd and $s \ge 2$. By Lemma \ref{lem7}, $u_n = u_{2k' 2^s \pm 2} \equiv (-1)^{k'} u_2 \pmod{u_{2^s}}$. Since $k'$ is odd
and $u_2 = 7$, it follows that $4M^2 = u_n-3 \equiv -10 \pmod{u_{2^s}}$ and, hence,
$$\left(\dfrac{-2}{u_{2^s}}\right) \left(\dfrac{5}{u_{2^s}}\right)
= \left(\dfrac{-10}{u_{2^s}}\right)
= \left(\dfrac{4M^2}{u_{2^s}}\right)
\cancel{= 1}
\changed{= \left(\dfrac{2M}{u_{2^s}}\right)^2 \ge 0}$$

By Lemma \ref{lem9} it follows that the first factor on the left is $1$. By Lemma \ref{lem8} it follows
that the second factor on the left is $-1$. Since this is impossible, we may conclude
that $n = 2$ and hence $u_n = 7$.
\end{proof}

Suppose now that $x$ is an odd positive integer and $x(x+1)(2x+1) = 6y^2$ for
some integer $y$. \changed{Since $x+1$ is even, we have $x((x+1)/2)(2x+1) = 3y^2$. Since $x$, $(x+1)/2$ and $2x+1$ are relatively prime in pairs, by Lemma \ref{lem3a}, we have $x \not\equiv 2 \pmod 3$ and $(x+1)/2 \not\equiv 2 \pmod 3$.} \sout{Since $x$, $x+1$ and $2x+1$ are relatively prime in pairs, it follows
that $x$ is either a square or a triple of a square, and hence $x \not\equiv 2 \pmod 3$. Moreover,
$x+1$, being even, is either double a square or six times a square, and, hence,
$x+1 \not\equiv 1 \pmod 3$.} Thus $x \equiv 1 \pmod 3$ and hence \sout{$x+1 \equiv 2 \pmod 3$ and} $2x+1 \equiv 0 \pmod 3$.
Thus for some nonnegative integers $u$, $v$ and $w$, we have $x = u^2$,
$x+1 = 2v^2$ and $2x+1 = 3w^2$. From this we obtain $6w^2+1 = 4x+3 = 4u^2+3$.
Also $(6w^2+1)^2-3(4vw)^2 = 12w^2(3w^2+1-4v^2)+1 = 12w^2(2x+1+1-2(x+1))+1 = 1$. \changed{Thus $u_n = 6w^2+1 = 4u^2+3$ for some $n$.}
Hence, by Lemma \ref{lem10}, \changed{$4u^2+3 = 7$} \sout{$6w^2+1 = 7$}. Thus \changed{$x = u^2 = 1$} \sout{$w = 1$ and
$x = 1$}. This gives us the trivial $1$ cannonball solution to Lucas's problem.

We may conclude that if a square number of cannonballs are stacked in a square
pyramid then there are exactly $4900$ of them.

\end{document}
